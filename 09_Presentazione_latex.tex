%sito Overleaf registrati e sciegli un modello (tesi/articolo/libro/etc)
%esempio_questa tipoligia di script inizia sempre con \ e raccolgono le funzioni nelle parentesi graffe
%mentre le parentesi quadre sono utilizzate quando si vuole modificare la struttura
%usepackage servono per pacchetti aggiuntivi 
% modelli per presentazioni : https://mpetroff.net/files/beamer-theme-matrix/
% pagina per funzioni dimensione testo: https://www.javatpoint.com/latex-fonts-size-and-styles

%PRESENTAZIONE NB: per R inserire un commento si utilizza il # mentre per LaTex si usa %

\documentclass{beamer}  #BEAMER è un pezzo di softuer che serve proprio per costruire delle presentazioni
\usepackage{graphicx} % Required for inserting images
\usepackage{url}
\usepackage{hyperref}
\usepackage{subcaption}
\usepackage{ragged2e}
\usepackage[T1]{fontenc}

\usetheme{Frankfurt}  % per decidere che colori e grafica usare nella presentazione
\usecolortheme{orchid}

\title{La mia prima presentazione LaTex 
- Fondamenti di Ecologia}
\author{Anna Slomp}
%\date{July 2024} %inserisce in automatico la data di oggi

% inizio della presentazione
\begin{document}
\maketitle
%caratteristica di rendere chiaro la struttura non in uso
\AtBeginSection[]
{
\begin{frame}
\frametitle{Outline}
\tableofcontents[currentsection]
\end{frame}
}

%usiamo sempre le sezioni per dividere gli argomenti in modo tale che risulti sempr eben ordinata:
\section{Introduction}
\begin{frame}{Prima slide} %aggiungiamo una nuova pagina e scriviamo il titolo
Questa è la mia prima slide.  %aggiungiamo il testo della prima slide  
\end{frame} %fine della mia pagina

%solitamente per le presentazioni si utilizzano elenchi puntati, seconda slide con elenco puntanto:
\begin{frame}{Elenchi puntati}
Ecology is based on the following frameworks:
\begin{itemize}
    \item Biology
    \item Geology
\end{itemize}    
\end{frame}
%NB se io voglio ricreare i punti man mano che proseguo nelle slide posso utilizzare \pause
%esempio: \item Biology
%         \item \pause Geology

%se vogliamo inserire delle frasi con parole in grassetto:
\begin{frame}{Elenchi puntati}
Ecology is based on the following frameworks:
\begin{itemize}
    \item Biology prima frase inserire parola in \textbf{grassetto}
    \item \pause Geology seconda frase inserire parola in \textbf{grassetto}
\end{itemize}    
\end{frame}

\section{Metodi}
\begin{frame}{Usiamo una formula} %ricordiamoci di latew wikibook
\begin{equation}
    E_c = 1/2 mv2 
% _ mi rende il "c" come pedice
% ^ mi permette di scrivere l'apice
% la funzione \frac {1}{2} mi permette di scrivere la frazione in modo corretto
%con la funzione \times inserisco la moltiplicazione
% quindi il risultato sarà: 
E_c = \frac{1}{2} \times m \times v^2
\end{equation}
\label{eq.cinetic} %per fare collegamento...
\end{frame}

\section{Risulati}
\begin{frame}{Grafico dei risultati} % vogliamo inserire un grafico salvato cul nostro pc
  \includegraphics[width = \textwidth]{Graphical-representation-Butterfly-effect.png}
  % width = \textwidth è necessario per rendere l'immagine grande come la schermata testo; se prima della funzione inserisco 0.7 ad             esempio questa si rimpicciolisce: width = 0.7\textwidth
  %per centrarla prima della funzione |begin(frame) inserisci la funzione \cettering

%...voglio inserire il collegamento alla formula
\begin{frame}{Discussione}
Come abbiamo visto nell'equazione \ref{eq.cinetic}.  %così mi scriverà il numero e lo aggiornerà se cambia  
\end{frame}

\end{document}
