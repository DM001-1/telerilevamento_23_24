\documentclass[10pt]{beamer}
\usepackage{graphicx} % Required for inserting images
\usepackage{url}
\usepackage{hyperref}
\usepackage{hyperref}
\usepackage{subcaption}
\usepackage{setspace}
\usepackage{ragged2e}
\usepackage[T1]{fontenc}

\usetheme{Frankfurt}  
\usecolortheme{orchid}
\sloppy

\title{Calamento (TN)\\
dal 2018 al 2023}
\author{anna slomp}
%\date{July 2024}

\begin  {document}
\maketitle


\AtBeginSection[]
{
\begin{frame}
\frametitle{Outline}
\tableofcontents[currentsection]
\end{frame}
}
\section{Introduzione}
\begin{frame}{Calamento, frazione di Telve (TN)}
Questo progetto vuole analizzare la perdita di vegetazione presso la località Calamento in seguito all'evento Vaia ed osservare lo sviluppo del bosco fino al 2023:
\begin{itemize}
    \item Inquadramento dell'area attraverso l'utilizzo\\
    di Sentinel 2 - Copernicus;
    \item Analisi NDVI and PCA; 
    \item Classificazione dell'area;
    \item Conclusioni;
\end{itemize}
Lo strumento utilizzato per l'analisi è R coding:
\href{https://github.com/DM001-1/telerilevamento_23_24/blob/main/R_script_exam.R}{Vedi link GitHub}
\end{frame}    

\section{Sentinel 2 - portale di Copernicus}
\begin{frame}{Raccolta immagini satelitari dell'area di interesse:}
Confronto temporale fra le immagini a colori naturali: 
\includegraphics[width = \textwidth]{immagini_satellite.jpeg}
\end{frame}

\begin{frame}{Copernicus - identificazione delle singole bande:}
\begin{figure}
        \centering
        \includegraphics[width=.4\linewidth]{nir_2018.jpeg}
        \includegraphics[width=.4\linewidth]{nir_2019.jpeg} 
        \includegraphics[width=.4\linewidth]{nir_2020.jpeg}
        \includegraphics[width=.4\linewidth]{nir_2021.jpeg}
        \includegraphics[width=.4\linewidth]{nir_2022.jpeg} 
        \includegraphics[width=.4\linewidth]{nir_2023.jpeg}
    \end{figure}
\end{frame}
\section{NDVI - Normalized Difference Vegetation Inde; \\
PCA - Principal Component Analysis}

\begin{frame}{NDVI}
Indicatore utilizzato per misurare la densità e la salute della vegetazione. Viene calcolato utilizzando la luce riflessa nelle bande del rosso e dell'infrarosso.
\begin{equation}
NDVI = \frac{(NIR-Red)}{(NIR+Red)} 
\end{equation}
\label{NDVI}
I valori di NDVI variano da -1 e +1: valori vicino a 1 indicano vegetazione densa e sana; zero indicano le rocce/sabbia/neve; i negativi indicano acqua o nuvole. 
\end{frame}

\begin{frame}{NDVI}
\begin{figure}
        \centering
        \includegraphics[width=.4\linewidth]{NDVI_2018.jpeg}
        \includegraphics[width=.4\linewidth]{NDVI_2019.jpeg} 
        \includegraphics[width=.4\linewidth]{NDVI_2020.jpeg}
        \includegraphics[width=.4\linewidth]{NDVI_2021.jpeg}
        \includegraphics[width=.4\linewidth]{NDVI_2022.jpeg} 
        \includegraphics[width=.4\linewidth]{NDVI_2023.jpeg}
    \end{figure}
\end{frame}

\begin{frame}{Multivariate Analysis - PCA}
\begin{itemize}
    \item Tecnica statistica che riassume le bande componenti un immagine, in un solo livello e e seleziona i componenti principali;
    \item PC1 prima componente principale, PC2 seconda componente principale, che catturano la maggior parte della varianza nei dati; 
    \item La visualizzazione in ggplot aiuta a capire la variazione del dataset dei componenti principali.
    \item Un valore maggiore di zero significa che la componente principale cattura una certa quantità di varianza dei dati originali, quindi questa componente è utile per rappresentare i dati; se fosse zero non ci sarebbe nessuna variazione dati.
\end{itemize}
\end{frame}

\begin{frame}
Differenza di vegetazione dal 2018 fino al 20219
\includegraphics[width = \textwidth]{PCA_NDVI_18_19.jpeg}
Più il valore si sposta verso 0.6 maggiore è la perdita di vegetazione.
\end{frame}
\begin{frame}
Differenza di vegetazione dal 2020 fino al 2023
\includegraphics[width = \textwidth]{PCA_NDVI_20_23.jpeg}
Più il valore si sposta verso -0.6 acquisizione di vegetazione.
\end{frame}   

\begin{frame}{Deviazione standard}
\begin{figure}
        \centering
        \includegraphics[width=.5\linewidth]{Deviazione standard 18_19.jpeg}
        \includegraphics[width=.5
        \linewidth]{deviazione standard_20_23.jpeg} 
    \end{figure}
      Consente di visualizzare la viariazione dei dati nel tempo e l'incerteza associata (margine di errore statistico).
\end{frame}

\section{Classificazione}
\begin{frame}{Classification}
Selezione, del numero di livelli energetici nell'immagine, per interpretare la copertura vegetale.
\includegraphics[width = \textwidth]{Classificazione_3.jpeg}
\end{frame}
\begin{frame}
Frequenza dei pixel di ogni classe:
\begin{figure}
        \centering
        \includegraphics[width=.4\linewidth]{grafico_class_18.jpeg}
        \includegraphics[width=.4\linewidth]{Gr_calss_19.jpeg} 
        \includegraphics[width=.4\linewidth]{gr_clas_20.jpeg}
        \includegraphics[width=.4\linewidth]{gr_clas_21.jpeg}
        \includegraphics[width=.4\linewidth]{gr_clas_22.jpeg} 
        \includegraphics[width=.4\linewidth]{gr_cals_23.jpeg}
    \end{figure}
\end{frame}

\section{Previsione NDVI per il 2024}
\begin{frame}{Andamento di crescita della vegetazione per l'anno 2024:}
\centering
\includegraphics[width = \textwidth]{Tendenza di crescita.jpeg}  
\end{frame}

\section{Conclusioni}
\begin{frame}{Conclusioni}
\begin{itemize}
    \item Dall'analisi multivariata PCA possiamo affermare che, in seguito all'evento Vaia dal 2018 al 2019, abbiamo una perdita misurabile di foresta, composta principalmente da \textit{Picea abies};
    \item In seguito all'evento possiamo misurare un leggero aumento del bosco in alcune aree all'interno dell'ambito di studio, ma resta comunque una percentuale di perdita di vegetazione;
\end{itemize}
\end{frame}

\begin{frame}{Conclusioni}
\centering
\includegraphics[width = \textwidth]{tabella classificazione.PNG} 
\begin{itemize}
    \item Si può ipotizzare che il continuo crescere della percentuale del suolo nudo anche dopo l'evento sia dovuto dal bostrico che attacca le piante rimaste in piedi da Vaia.
     \item Si ipotizza che l'andamento dell'indice NDVI per l'anno 2024 sia verso l'apertura di nuovi spazi all'interno del bosco, quindi in continua perdita di bosco. 
\end{itemize} 
\end{frame}

\end{document}
