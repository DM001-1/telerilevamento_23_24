\documentclass[10pt]{beamer}
\usepackage{graphicx} % Required for inserting images
\usepackage{url}
\usepackage{hyperref}
\usepackage{hyperref}
\hypersetup{colorlinks = true}
\usepackage{subcaption}
\usepackage{setspace}
\usepackage{ragged2e}
\usepackage[T1]{fontenc}

\usetheme{Frankfurt}  
\usecolortheme{orchid}

\title{Calamento (TN)\\
dal 2018 al 2023}
\author{Anna Slomp}
\date{July 2024}

\begin  {document}
\maketitle


\AtBeginSection[]
{
\begin{frame}
\frametitle{Outline}
\tableofcontents[currentsection]
\end{frame}
}
\section{Introduzione}
\begin{frame}{Calamento, frazione di Telve (TN)}
Questo progetto vuole analizzare la perdita di vegetazione presso la località Calamento in seguito all'evento Vaia ed osservare lo sviluppo del bosco fino al 2023:
\begin{itemize}
    \item Inquadramento dell'area attraverso l'utilizzo\\
    di Sentinel 2 - Copernicus;
    \item Analisi NDVI and PCA; 
    \item Classificazione dell'area;
    \item Conclusioni;
\end{itemize}
Lo strumento utilizzato per l'analisi è R coding:
\href{https://github.com/DM001-1/telerilevamento_23_24/blob/main/R_script_exam.R}{Vedi link GitHub}
\end{frame}    

\section{Sentinel 2}
\begin{frame}{Raccolta immagini satelitari dell'area di interesse:}
Confronto temporale fra le immagini a colori naturali: 
\includegraphics[width = \textwidth]{immagini_satellite.jpeg}
\end{frame}

\begin{frame}{Copernicus - identificazione delle singole bande}
\begin{figure}
La vegetazione riflette la radiazione NIR in modo molto efficiente:
        \centering
        \includegraphics[width=.3\linewidth]{nir_2018.jpeg}
        \includegraphics[width=.3\linewidth]{nir_2019.jpeg} 
        \includegraphics[width=.3\linewidth]{nir_2020.jpeg}
        \includegraphics[width=.3\linewidth]{nir_2021.jpeg}
        \includegraphics[width=.3\linewidth]{nir_2022.jpeg} 
        \includegraphics[width=.3\linewidth]{nir_2023.jpeg}
Comando R per impostare la banda degli infrarossi (NIR) al posto della banda del rosso e riuscire a visualizzarli:\\ 
> im.plotRGB(rgb, r=4, g=3, b=2)
    \end{figure}
\end{frame}


\section{NDVI - PCA}
\begin{frame}{NDVI - standardizzazione dei dati}
Indicatore utilizzato per misurare la densità e la salute della vegetazione. Viene calcolato utilizzando la luce riflessa nelle bande del rosso e dell'infrarosso.
\begin{equation}
NDVI = \frac{(NIR-Red)}{(NIR+Red)} 
\end{equation}
\label{NDVI}
I valori di NDVI variano da -1 e +1: valori vicino a 1 indicano vegetazione densa e sana; zero indicano le rocce/sabbia/neve; i negativi indicano acqua o nuvole. 
\end{frame}

\begin{frame}{NDVI}
\begin{figure}
        \centering
        \includegraphics[width=.4\linewidth]{NDVI_2018.jpeg}
        \includegraphics[width=.4\linewidth]{NDVI_2019.jpeg} 
        \includegraphics[width=.4\linewidth]{NDVI_2020.jpeg}
        \includegraphics[width=.4\linewidth]{NDVI_2021.jpeg}
        \includegraphics[width=.4\linewidth]{NDVI_2022.jpeg} 
        \includegraphics[width=.4\linewidth]{NDVI_2023.jpeg}
    \end{figure}
\end{frame}

\begin{frame}{Multivariate Analysis - PCA}
\begin{itemize}
    \item Tecnica statistica che riassume le bande componenti un immagine, in un solo livello e seleziona i componenti principali; 
    \item Un valore maggiore di zero significa che la componente principale cattura una certa quantità di varianza dei dati originali, quindi questa componente è utile per rappresentare i dati; se fosse zero non ci sarebbe nessuna variazione dati.
    \item Visualizzazione PCA dell'indice di vegetazione.
\end{itemize}
\end{frame}

\begin{frame}{Differenza di vegetazione dal 2018 fino al 2019}
\centering
\includegraphics[width = .8\textwidth]{PCA_NDVI_18_19.jpeg}\\
Più il valore si sposta verso 0.6 maggiore è la perdita di vegetazione.\\
Si evidenziano le zone più colpite.
\end{frame}
\begin{frame}{Differenza di vegetazione dal 2020 fino al 2023}
\centering
\includegraphics[width = .8\textwidth]{PCA_NDVI_20_23.jpeg}\\
Più il valore si sposta verso -0.6 acquisizione di vegetazione.\\
L'analisi PCA indica che le variazioni osservate  sono statisticamente significative suggerendo un cambiamento reale nella vegetazione più tosto che flutuazioni casuali dei dati.
\end{frame}   

\begin{frame}{Deviazione standard}
\begin{figure}
        \centering
        \includegraphics[width=.4\linewidth]{Deviazione standard 18_19.jpeg}
        \includegraphics[width=.4
        \linewidth]{deviazione standard_20_23.jpeg} 
    \end{figure}
      \item Consente di visualizzare la viariazione dei dati nel tempo e l'incerteza associata: maggior variabilità, causa di eventi estremi; minor variabilità, maggior stabilità della vegetazione.\\
      Comando R: \\
      > sd <- focal (pci1 [[1]], matrix(1/9, 3, 3), fun = sd)\\
      > sdf <- as.data.frame(sd, xy = T)
\end{frame}


\section{Classificazione}
\begin{frame}{Classification}
Selezione, del numero di livelli energetici nell'immagine, per interpretare la copertura vegetale.
\includegraphics[width = \textwidth]{Classificazione_3.jpeg}
\end{frame}
\begin{frame}
Frequenza dei pixel di ogni classe:
\begin{figure}
        \centering
        \includegraphics[width=.35\linewidth]{grafico_class_18.jpeg}
        \includegraphics[width=.35\linewidth]{Gr_calss_19.jpeg} 
        \includegraphics[width=.35\linewidth]{gr_clas_20.jpeg}
        \includegraphics[width=.35\linewidth]{gr_clas_21.jpeg}
        \includegraphics[width=.35\linewidth]{gr_clas_22.jpeg} 
        \includegraphics[width=.35\linewidth]{gr_cals_23.jpeg}\\
        \includegraphics[width = 0.35 \textwidth]{tabella classificazione.PNG}
    \end{figure}
\end{frame}

\section{Previsione NDVI per il 2024}
\begin{frame}{Andamento di crescita della vegetazione per l'anno 2024 - Calcolo R }
\centering
\includegraphics[width = 0.8 \textwidth]{Codici.png}
\end{frame}
\begin{frame}{Andamento di crescita della vegetazione per l'anno 2024 - Grafico}
\centering
\includegraphics[width = 0.8 \textwidth]{Tendenza di crescita.jpeg}
\end{frame}

\section{Conclusioni}
\begin{frame}{Conclusioni}
Si può ipotizzare che: 
\begin{itemize}
    \item il continuo diminuire della percentuale che identifica la foresta di conifere \textit{Picea Abies}, anche dopo l'evento, sia dovuto dal bostrico; questo coleottero se presente in numero elevato può attaccare anche le piante rimaste in piedi da Vaia; 
     \item osservando l'indice NDVI per l'anno 2024 si può vedere come questo si sposti verso valori positivi. Inoltre troviamo, in alcune aree, una certa stabilità di vegetazione. Questo potrebbe indicare una sostituzione con altri tipi di vegetazione visto la continua diminuzione della percentuale di foresta.
     \item questi risultati evidenziano l'importanza di monitorare e gestire attentamente le risorse forestali per mitigare gli effetti futuri come l'espasione di infestanti, la variazione della biodiversità e la variazione dell'ecosistema dell'area di Calamento. 
\end{itemize} 
\end{frame}

\end{document}
